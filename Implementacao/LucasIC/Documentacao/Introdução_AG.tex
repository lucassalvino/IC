\documentclass[12pt,twoside,openright,a4paper,brazil]{article}
\usepackage[brazilian]{babel}
\usepackage[utf8]{inputenc}
\usepackage{lipsum}
\usepackage[usenames]{color}
\title{Introdução à Algoritmos Genéticos}
\author{Lucas Salvino de Deus \\ lucassalvino1@gmail.com}
\date{\today}


\usepackage{indentfirst}
\begin{document}

	\begin{titlepage}%página de título e autor
	\maketitle
	\end{titlepage}
	
	\begin{abstract}%resumo do trabalho
		Resumo de fundamentos e termos à Algoritmos Genéticos (AG). Será apresentado no dia primeiro de Setembro de 2017, como introdução aos estudos do projeto. Apresenta os principais conceitos e sujestões de bibliografia para estudo no decorrer do projeto.
		
		Baseado nos conceitos de seleção natural apresentados por Darwin em 1858, os Algoritmos Genéticos (AG) são uma técnica de busca eficiente em varrer o espaço de soluções e em encontrar soluções próximas à ótima.
	\end{abstract}
	
	\section{Introdução}
		Métodos de busca convencionais, são muito utilizados por apresentar desempenho satisfatório para conjunto de dados finitos e onde se conhece a estrutura onde os dados estão armazenados. Para estruturas pouco conhecida, dados dinâmicos ou grande quantidades de informações envolvidas em cada busca, os métodos convêncionais começam a apresentar problemas de desempenho. Os AG são utilizados para esses casos.
		
		Os AG são um técinca computacional utilizada para encontrar soluções aproximadas em problemas de otimização e busca. Seu funcionamento é totalemente baseada na teoria da Seleção Natural proposta por Darwin.
		
		Segunda a mesma,as espécies atuais evoluiram de variações de suas gerações antecessoras e ainda continuam a se alterar. Em resumo, variações vantajosas apresentadas por um índivíduo possuem mais chances de serem mantidas em sua população, com o passar das gerações, essas carcteristicas passam a ser comuns para a população, enquanto as demais variações menos 'benéficas' ao índivíduo (e à população) não conseguem prevalecer para as gerações futuras.Essas alterações ao decorrer de várias gerações acabaram por gerar toda a vida que conhecemos atualmente.
			
	\section{Definições}
	\begin{description}
		\item[Gene] Caracteristica isolada de um indivíduo, em genética é a unidade fundamental da hereditariedade.
		\item[Cromossomos] Conjunto de genes que representão todas as caracteristicas de um indivíduo, em genética, é um conjunto de DNA, que por sua vez são formadas de genes.
		\item[População] Conjunto de índivíduos, em biologia genética, é um ramo que estuda a mudança na frequência de alelos sob influência das quatro forças evolutivas: seleção naturalm deriva gênica, mutação e fluxo gênico.
		\item[Geração] Cada Ciclo do algoritmo, espaço de tempo que separa cada um dos graos de filiação.
		\item[Cruzamentento] Processo no qual se gera filhos a partir de dois indivíduos, acasalento entre indivíduos distintos.
		\item[Seleção] Processo no qual os índivíduos mais adaptaveis ao ambiente conseguem sobreviver por mais tempo ou atrair mais atenção de outras índividuos da população, garantido sua reprodução e assim passando seus cromossomos para as gerações futuras.
		\item[Mutação] Erros que ocorrem durante a cópia de cromossomos, esses erros podem levar a criação de novas caracteristicas, essas podem ser vantajosas para os novos indivíduos, ou prejudiciais.
	\end{description}

	\subsection{Princípios}
		\begin{itemize}
		\item Indivíduos disputam por recursos no ambiente.
		\item Indivíduos mais adaptados possuem uma maior probabilidade de sobreviver e consequente, se reproduzir
		\item O processo de evolução não ocorre sem problemas, durante o processo de cópia de um cromossomo, pode ocorrer erros, que são conhecidos como multação
		\end{itemize}
		
	\section{Algoritmo Genético}
	O Algoritmo genético simula todo o processo de seleção natural dentro de um modelo computacional. Geralmente um cromossomo é representado por um vetor, onde cada elemento representa um gene. Os genes assim como na genética representam uma caracteristica da solução, logo, cada cromossomo representa uma possível solução.
	
	Os passos para execução de um Algoritmo Genético segue os passos abaixo:
	\begin{enumerate}
	\item \textbf{População inicial}: A população inicial é gerada aleatriamente, mas desde que cada índividou (cromossomo) represente uma possível solução. 
	\end{enumerate}
	
\end{document}